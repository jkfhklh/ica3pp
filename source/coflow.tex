\documentclass[10pt, conference, letterpaper]{IEEEtran}
%\IEEEoverridecommandlockouts
% The preceding line is only needed to identify funding in the first footnote. If that is unneeded, please comment it out.
\usepackage{cite}
\usepackage{amsmath,amssymb,amsfonts}
\usepackage{algorithmic}
\usepackage{graphicx}
\usepackage{textcomp}
\usepackage{xcolor}
\def\BibTeX{{\rm B\kern-.05em{\sc i\kern-.025em b}\kern-.08em
    T\kern-.1667em\lower.7ex\hbox{E}\kern-.125emX}}
\begin{document}

\title{Coflow}

\author{Zifan Liu, Haipeng Dai and Wanchun Dou\\
State Key Laboratory for Novel Software Technology, Nanjing University, Nanjing, Jiangsu, China\\
zifanliu@smail.nju.edu.cn, haipengdai,douwh@nju.edu.cn}

\maketitle

\begin{abstract}
This document is a model and instructions for \LaTeX.
This and the IEEEtran.cls file define the components of your paper [title, text, heads, etc.]. *CRITICAL: Do Not Use Symbols, Special Characters, Footnotes,
or Math in Paper Title or Abstract.
\end{abstract}

\begin{IEEEkeywords}
keyword, keyword, keyword
\end{IEEEkeywords}

\section{Introduction}
This document is a model and instructions for \LaTeX.\cite{HUG}
Please observe the conference page limits.

\section{Model and Objective}
In this section, we describe the model of datacenter networks and coflow.
\subsection{Model}
To simplify the discussion, key terms used in our model are summarized in Table~1.
\begin{table}
\caption{Key Terms and Descriptions}
\begin{center}
\begin{tabular}{l@{\quad}l}
\hline
Terms & Description\\
\hline\rule{0pt}{12pt}
$S$ & The set of the points in device activity area.\\
$DA$ & The set of device activity area, $DA = \left\{da_{1}, da_{2},\dots, da_{N}\right\}$.\\
$da_{n}$ & The $n$--th device activity area.\\
$MD$ & The set of mobile devices, $MD = \left\{md_{1}, md_{2},\dots, md_{M}\right\}$.\\
$md_{n}$ & The set of mobile devices in $da_{n}, md_{n} = \left\{md_{n,1}, md_{n,2},\dots, md_{n,Z}\right\}$.\\
$mp_{m}$ & The position $md_{m}, mp_{m} = \left(mpx_{m}, mpy_{m}\right)$.\\
$cl_{n,i}$ & The $i$-th cloudlet in dan.\\
$cp_{n,i}$ & The central position of $cl_{n,i}, cp_{n,i} = \left(cpx_{n,i}, cpy_{n,i}\right)$.\\
$dc_{n,i}$ & The device collection of $cl_{n,i}$.\\
$r_{n,i}$ & The coverage radius for $cl_{n,i}$.\\
$IC_{n,i}$ & The indoor cloudlet coverage collection of $cl_{n,i}$.\\
$\rho$ & The density threshold for cloudlet placement judgment.\\[2pt]
\hline
\end{tabular}
\end{center}
\end{table}


\section*{Acknowledgment}



\section*{References}


\bibliographystyle{IEEEtran}
\bibliography{IEEEfull,myref}

%\begin{thebibliography}{00}
%\bibitem{b1} G. Eason, B. Noble, and I. N. Sneddon, ``On certain integrals of Lipschitz-Hankel type involving products of Bessel functions,'' Phil. Trans. Roy. Soc. London, vol. A247, pp. 529--551, April 1955.
%\bibitem{b2} J. Clerk Maxwell, A Treatise on Electricity and Magnetism, 3rd ed., vol. 2. Oxford: Clarendon, 1892, pp.68--73.
%\bibitem{b3} I. S. Jacobs and C. P. Bean, ``Fine particles, thin films and exchange anisotropy,'' in Magnetism, vol. III, G. T. Rado and H. Suhl, Eds. New York: Academic, 1963, pp. 271--350.
%\bibitem{b4} K. Elissa, ``Title of paper if known,'' unpublished.
%\bibitem{b5} R. Nicole, ``Title of paper with only first word capitalized,'' J. Name Stand. Abbrev., in press.
%\bibitem{b6} Y. Yorozu, M. Hirano, K. Oka, and Y. Tagawa, ``Electron spectroscopy studies on magneto-optical media and plastic substrate interface,'' IEEE Transl. J. Magn. Japan, vol. 2, pp. 740--741, August 1987 [Digests 9th Annual Conf. Magnetics Japan, p. 301, 1982].
%\bibitem{b7} M. Young, The Technical Writer's Handbook. Mill Valley, CA: University Science, 1989.
%\end{thebibliography}
%\vspace{12pt}
%\color{red}
%IEEE conference templates contain guidance text for composing and formatting conference papers. Please ensure that all template text is removed from your conference paper prior to submission to the conference. Failure to remove the template text from your paper may result in your paper not being published.

\end{document}
